%\documentclass[12pt,letterpaper]{article}\usepackage{float,amsmath,dcolumn,multicol,latexsym,ifthen,natbib,amssymb,verbatim,hyperref,vmargin,moreverb,cancel}\begin{document}

% Generated by section SSDerivations in the main text;
The steady state of the perfect foresight representative agent model occurs at the point
where
\begin{eqnarray}
  \util^{\prime}(c) & = & \beta \daleth \RGross  \Lambda^{-\CRRA} \util^{\prime}(c)
\\ \RGross  & = & \PGro^{\CRRA}/\beta \daleth
\\ 1+\kapShare k^{\kapShare-1} & = & \PGro^{\CRRA}/\beta \daleth
\\ k & = & \left((\PGro^{\CRRA}/\beta \daleth -1)/\kapShare\right)^{\frac{1}{\kapShare-1}}
\end{eqnarray}
and defining the intertemporal interest rate as $d M_{t+1}/d A_{t} = \RGross \daleth = \Rfree$,
if wages grow from the current level $\Wage$ by a factor $\PGro$
each year then the ratio of human wealth to current wages will be
\begin{eqnarray}
  h & = & \left(\frac{1}{1-\PGro/\Rfree}\right)
\end{eqnarray}
and the partial equilibrium perfect foresight model's marginal propensity to consume
(note that this neglects the effect of consumption on the interest rate) is
\begin{eqnarray}
  \MPC & = & (1-(\Rfree\beta)^{1/\CRRA}/\Rfree)
\end{eqnarray}
leading to a steady-state level of consumption of
\begin{eqnarray}
  c & = & (k\Rfree +h \Wage)\MPC
\\ & = & \left(m+(h-1)\Wage\right)\MPC
\end{eqnarray}
where in the second expression 1 must be subtracted from $h$ to
reflect the fact that current labor income $\Wage$ has
already been incorporated into $m$ and must not be double-counted
(as it otherwise would be since it was included in the infinite
sum that led to the expression for $h$).

%\end{document}

